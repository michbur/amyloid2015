\documentclass[10pt]{beamer}\usepackage[]{graphicx}\usepackage[]{color}
%% maxwidth is the original width if it is less than linewidth
%% otherwise use linewidth (to make sure the graphics do not exceed the margin)
\makeatletter
\def\maxwidth{ %
  \ifdim\Gin@nat@width>\linewidth
    \linewidth
  \else
    \Gin@nat@width
  \fi
}
\makeatother

\definecolor{fgcolor}{rgb}{0.345, 0.345, 0.345}
\newcommand{\hlnum}[1]{\textcolor[rgb]{0.686,0.059,0.569}{#1}}%
\newcommand{\hlstr}[1]{\textcolor[rgb]{0.192,0.494,0.8}{#1}}%
\newcommand{\hlcom}[1]{\textcolor[rgb]{0.678,0.584,0.686}{\textit{#1}}}%
\newcommand{\hlopt}[1]{\textcolor[rgb]{0,0,0}{#1}}%
\newcommand{\hlstd}[1]{\textcolor[rgb]{0.345,0.345,0.345}{#1}}%
\newcommand{\hlkwa}[1]{\textcolor[rgb]{0.161,0.373,0.58}{\textbf{#1}}}%
\newcommand{\hlkwb}[1]{\textcolor[rgb]{0.69,0.353,0.396}{#1}}%
\newcommand{\hlkwc}[1]{\textcolor[rgb]{0.333,0.667,0.333}{#1}}%
\newcommand{\hlkwd}[1]{\textcolor[rgb]{0.737,0.353,0.396}{\textbf{#1}}}%

\usepackage{framed}
\makeatletter
\newenvironment{kframe}{%
 \def\at@end@of@kframe{}%
 \ifinner\ifhmode%
  \def\at@end@of@kframe{\end{minipage}}%
  \begin{minipage}{\columnwidth}%
 \fi\fi%
 \def\FrameCommand##1{\hskip\@totalleftmargin \hskip-\fboxsep
 \colorbox{shadecolor}{##1}\hskip-\fboxsep
     % There is no \\@totalrightmargin, so:
     \hskip-\linewidth \hskip-\@totalleftmargin \hskip\columnwidth}%
 \MakeFramed {\advance\hsize-\width
   \@totalleftmargin\z@ \linewidth\hsize
   \@setminipage}}%
 {\par\unskip\endMakeFramed%
 \at@end@of@kframe}
\makeatother

\definecolor{shadecolor}{rgb}{.97, .97, .97}
\definecolor{messagecolor}{rgb}{0, 0, 0}
\definecolor{warningcolor}{rgb}{1, 0, 1}
\definecolor{errorcolor}{rgb}{1, 0, 0}
\newenvironment{knitrout}{}{} % an empty environment to be redefined in TeX

\usepackage{alltt}
\usetheme{Warsaw}
\usepackage{graphicx}
\usepackage[utf8]{inputenc}
\usepackage{amsfonts}
\usepackage{amsmath}
\usepackage[notocbib]{apacite}


\setbeamertemplate{caption}{\centering\insertcaption\par}
\setlength{\belowcaptionskip}{15pt}
\renewcommand{\thetable}{}

\providecommand{\e}[1]{\ensuremath{\times 10^{#1}}}
\IfFileExists{upquote.sty}{\usepackage{upquote}}{}
\begin{document}


\date{}
\author{Micha\l{}  Burdukiewicz\inst{1}, Piotr Sobczyk\inst{2}, }
\institute{ 
\inst{1} University of Wroc\l{}aw, Department of Genomics, Poland
\and 
\inst{2} Wroc\l{}aw University of Technology, Institute of Mathematics and Computer Science, Poland}

\title{n-grams (k-mers) and hidden semi_Markov models}


\AtBeginSection[]
{
\begin{frame}<beamer>
\frametitle{Outline}
\tableofcontents[currentsection]
\end{frame}
}



\section{n-grams (k-mers)}

\begin{frame}
n-grams (k-tuples) are vectors of n characters derived from input sequence(s). They may form continuous sub-sequences or be discontinuous.  


Important n-gram parameter is its position. Instead of just counting n-grams, one may want to count how many n-grams occur at a given position in multiple (e.g. related) sequences.


\end{frame}



\begin{frame}


% latex table generated in R 3.2.0 by xtable 1.7-4 package
% Sun May 03 22:38:45 2015
\begin{table}[ht]
\centering
\begin{tabular}{rllllll}
  \hline
 & P1 & P2 & P3 & P4 & P5 & P6 \\ 
  \hline
S1 & C & T & T & A & G & C \\ 
  S2 & C & A & G & A & C & G \\ 
  S3 & G & T & G & A & T & T \\ 
   \hline
\end{tabular}
\caption{Sample sequences.  S - sequence, P - position.} 
\end{table}



  
% latex table generated in R 3.2.0 by xtable 1.7-4 package
% Sun May 03 22:38:45 2015
\begin{table}[ht]
\centering
\begin{tabular}{rrrrr}
  \hline
 & A & C & G & T \\ 
  \hline
S1 & 1 & 2 & 1 & 2 \\ 
  S2 & 2 & 2 & 2 & 0 \\ 
  S3 & 1 & 0 & 2 & 3 \\ 
   \hline
\end{tabular}
\caption{1-gram counts.} 
\end{table}


\end{frame}

\begin{frame}



% latex table generated in R 3.2.0 by xtable 1.7-4 package
% Sun May 03 22:38:46 2015
\begin{table}[ht]
\centering
\begin{tabular}{rllllll}
  \hline
 & P1 & P2 & P3 & P4 & P5 & P6 \\ 
  \hline
S1 & C & T & T & A & G & C \\ 
  S2 & C & A & G & A & C & G \\ 
  S3 & G & T & G & A & T & T \\ 
   \hline
\end{tabular}
\caption{Sample sequences.  S - sequence, P - position.} 
\end{table}

  

  
% latex table generated in R 3.2.0 by xtable 1.7-4 package
% Sun May 03 22:38:46 2015
\begin{table}[ht]
\centering
\begin{tabular}{rrrrrrrrrrrrrrrrr}
  \hline
 & AA & CA & GA & TA & AC & CC & GC & TC & AG & CG & GG & TG & AT & CT & GT & TT \\ 
  \hline
S1 & 0 & 0 & 0 & 1 & 0 & 0 & 1 & 0 & 1 & 0 & 0 & 0 & 0 & 1 & 0 & 1 \\ 
  S2 & 0 & 1 & 1 & 0 & 1 & 0 & 0 & 0 & 1 & 1 & 0 & 0 & 0 & 0 & 0 & 0 \\ 
  S3 & 0 & 0 & 1 & 0 & 0 & 0 & 0 & 0 & 0 & 0 & 0 & 1 & 1 & 0 & 1 & 1 \\ 
   \hline
\end{tabular}
\caption{2-gram counts.} 
\end{table}


\end{frame}


\end{document}
